\documentclass[12pt,a4paper]{article}
\usepackage[english, ukrainian]{babel}
\usepackage[utf8]{inputenc}
\usepackage[T2A]{fontenc}
\usepackage[left=2.5cm,top=2cm,right=1.5cm,bottom=2cm,nohead]{geometry}
\usepackage{setspace}
\usepackage{listings}
\usepackage{color}
\usepackage{float}
\usepackage[pdftex]{graphicx}
\usepackage{courier}
\usepackage{bold-extra}
\usepackage{fix-cm}
\usepackage{alltt}
\usepackage{indentfirst}
\usepackage{amsmath, amsthm, amssymb}
\usepackage{url}

\setstretch{1.1}

\begin{document}
\pretolerance=-1
\tolerance=2300

\thispagestyle{empty}
\setlength{\parindent}{1.5cm}
\fontsize{14pt}{6mm}\selectfont

\begin{center}
  Міністерство освіти і науки, молоді та спорту України
  
  Львівський національний університет імені Івана Франка

  Факультет прикладної математики та інформатики
\end{center}


\vspace{6cm}

\begin{center}
  {\bfseries\Large Реферат}\\[0.5cm]
  на тему:\\[0.5cm]
  {\bfseries\Large <ТЕМА>}\\
\end{center}

\vspace{2cm}

\begin{flushleft}\leftskip11cm
  Виконав\\
  студент <ГРУПА>\\
  <ПРИЗВІЩЕ ІМ'Я>
\end{flushleft}

\vspace{6cm}

\begin{center}
  Львів - <РІК> 
\end{center}

\clearpage

\setstretch{1.5}
\fontsize{14pt}{6mm}\selectfont

\newcommand{\vect}[1]{(#1_1,#1_2,\dots,#1_n)}

\thispagestyle{empty}
\tableofcontents
\clearpage
\pagenumbering{arabic}

\section{Вступ}

Лорем іпсум!\cite{alias}

\clearpage

\section{<РОЗДІЛ \#>}

І так далі\cite{web}

\clearpage
\addcontentsline{toc}{section}{Література}
\begin{thebibliography}{9}

  \bibitem{alias}<АВТОР> \emph{<КНИГА>},
    <ВИДАВНИЦТВО> <РІК>, <К-КІСТЬ СТОРІНОК> ст.
    
  \bibitem{web}<АВТОР> \emph{<НАЗВА>} [Електронний ресурс],
    <РІК>. Режим доступу:
    \url{https://github.com/Uko/thesis-template}

\end{thebibliography}

\end{document}
