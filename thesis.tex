\documentclass[12pt,a4paper]{article}
\usepackage[english, ukrainian]{babel}
\usepackage[utf8]{inputenc}
\usepackage[T2A]{fontenc}
\usepackage[left=2.5cm,top=2cm,right=1.5cm,bottom=2cm,nohead]{geometry}
\usepackage{setspace}
\usepackage{listings}
\usepackage{color}
\usepackage{float}
\usepackage[pdftex]{graphicx}
\usepackage{courier}
\usepackage{bold-extra}
\usepackage{fix-cm}
\usepackage{alltt}
\usepackage{indentfirst}
\usepackage{amsmath, amsthm, amssymb}
\usepackage{url}

\setstretch{1.1}
\pagestyle{empty}

\begin{document}
\pretolerance=-1
\tolerance=2300

\setlength{\parindent}{1.5cm}
\fontsize{14pt}{6mm}\selectfont

\begin{center}
  Міністерство освіти і науки, молоді та спорту України
  
  Львівський національний університет імені Івана Франка

  Факультет прикладної математики та інформатики
\end{center}


\vspace{6cm}

\begin{center}
  {\bfseries\Large Реферат}\\[0.5cm]
  на тему:\\[0.5cm]
  {\bfseries\Large Метод послідовних наближень}\\
\end{center}

\vspace{2cm}

\begin{flushleft}\leftskip11cm
  Виконали\\
  студенти ПМІ-51м\\
  Михалевич Ігор\\
  Тимчук Юрій
\end{flushleft}

\vspace{6cm}

\begin{center}
  Львів - 2012 
\end{center}

\clearpage

\setstretch{1.5}
\fontsize{14pt}{6mm}\selectfont

\section{Сталі Шварца}
Метод послідовних наближень можна застосовувати при дуже загальних припущеннях, а саме, якщо власне значення $\lambda$ входить лінійно в диференціальне рівняння і крайові умови (процес можна застосовувати також для диференціальних рівнянь в часткових похідних). В основу будуть покладені диференціальні рівняння (\emph{TODO}) та крайові умови (\emph{TODO}).

В загальному випадку метод полегає в наступному. Починаючи з довільно вибраної функції $F_0(x)$, визначають послідовність функцій $F_1, F_2, \dots$ при цьому $F_k$ отримують із $F_{k+1}$, шляхом розв'язку крайової задачі. У всіх членах диференціального рівняння и крайових умов, які містять $\lambda$ в якості множника, $\lambda y$ щаміняють на $F_{k-1}$, а в членах, вільних від $\lambda$, заміняють $y$ на $F_k$. Наприклад, для задачі $- y'' = \lambda y; y(1) = 0; \lambda y'(0) = y(0)$ отрамали б
\[
	- F''_k = F_{k-1}; F_k(1) = 0; F'_{k-1}(0) = F_k(0).
\] 
Процес спрощується, якщо $\lambda$ не входить в крайові умови. Тоді, починаючи із $F_0(x)$, функції $F_1, F_2, \dots$ визначають шдяхом розв'язку крайових задач
\[
\left.
\begin{array}{l}
	M[F_k] = N[F_{k-1}],\\
	U_\mu[F_k] = 0
\end{array} \right\}
\quad
(k = 1, 2, \dots).
\]
Можна очікувати, що у випадку збіжності $F_n$ з ростом $n$ все більше приймають вигляд власної функції $y_s$. Тоді $\frac{M[F_n]}{N[F_n]}$ повинне бути приблизно рівним відповідному власному значенню $\lambda_s$. Це відношення залежить ще й від $x$.

\end{document}
