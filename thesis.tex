\documentclass[12pt,a4paper]{article}
\usepackage[english, ukrainian]{babel}
\usepackage[utf8]{inputenc}
\usepackage[T2A]{fontenc}
\usepackage[left=2.5cm,top=2cm,right=1.5cm,bottom=2cm,nohead]{geometry}
\usepackage{setspace}
\usepackage{listings}
\usepackage{color}
\usepackage{float}
\usepackage[pdftex]{graphicx}
\usepackage{courier}
\usepackage{bold-extra}
\usepackage{fix-cm}
\usepackage{alltt}
\usepackage{indentfirst}
\usepackage{amsmath, amsthm, amssymb}
\usepackage{url}

\setstretch{1.1}
\pagestyle{empty}

\begin{document}
\pretolerance=-1
\tolerance=2300

\setlength{\parindent}{1.5cm}
\fontsize{14pt}{6mm}\selectfont

\begin{center}
  Міністерство освіти і науки, молоді та спорту України
  
  Львівський національний університет імені Івана Франка

  Факультет прикладної математики та інформатики
\end{center}


\vspace{6cm}

\begin{center}
  {\bfseries\Large Реферат}\\[0.5cm]
  на тему:\\[0.5cm]
  {\bfseries\Large Метод послідовних наближень}\\
\end{center}

\vspace{2cm}

\begin{flushleft}\leftskip11cm
  Виконали\\
  студенти ПМІ-51м\\
  Михалевич Ігор\\
  Тимчук Юрій
\end{flushleft}

\vspace{6cm}

\begin{center}
  Львів - 2012 
\end{center}

\clearpage

\setstretch{1.5}
\fontsize{14pt}{6mm}\selectfont

\section{Сталі Шварца}




\end{document}
