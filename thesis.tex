\documentclass[12pt,a4paper]{article}
\usepackage[english, ukrainian]{babel}
\usepackage[utf8]{inputenc}
\usepackage[T2A]{fontenc}
\usepackage[left=2.5cm,top=2cm,right=1.5cm,bottom=2cm,nohead]{geometry}
\usepackage{setspace}
\usepackage{listings}
\usepackage{color}
\usepackage{float}
\usepackage[pdftex]{graphicx}
\usepackage{courier}
\usepackage{bold-extra}
\usepackage{fix-cm}
\usepackage{alltt}
\usepackage{indentfirst}
\usepackage{amsmath, amsthm, amssymb}
\usepackage{url}

\setstretch{1.1}
\pagestyle{empty}

\begin{document}
\pretolerance=-1
\tolerance=2300

\setlength{\parindent}{1.5cm}
\fontsize{14pt}{6mm}\selectfont

\begin{center}
  Міністерство освіти і науки, молоді та спорту України
  
  Львівський національний університет імені Івана Франка

  Факультет прикладної математики та інформатики
\end{center}


\vspace{6cm}

\begin{center}
  {\bfseries\Large Реферат}\\[0.5cm]
  на тему:\\[0.5cm]
  {\bfseries\Large Метод послідовних наближень}\\
\end{center}

\vspace{2cm}

\begin{flushleft}\leftskip11cm
  Виконали\\
  студенти ПМІ-51м\\
  Михалевич Ігор\\
  Тимчук Юрій
\end{flushleft}

\vspace{6cm}

\begin{center}
  Львів - 2012 
\end{center}

\clearpage

\setstretch{1.5}
\fontsize{14pt}{6mm}\selectfont

\section{Вступ}

Нехай диференціальне рівняння має такий вигляд
\[
T[u] = \lambda S[u]   \quad (4.11)
\]
де $Tu$ та $Su$ лінійні однорідні звичайні диференціальні вирази типу
\[
T[u] = \sum_{v=0}^t(-1)^v[f_v(x)u^{(v)}(x)]^{(v)}   \quad (4.12)
\]
\[
Su = \sum_{v=0}^s(-1)^v[g_v(x)u^{(v)}(x)]^{(v)}   \quad (4.13)
\]
Тут $(f_v(x)$ та $g_v(x)$ задані, суттєві, $v$-катно неперервно диференційовні функції. 

Нехай $t>s$ тоді, відповідно, $2t$ -- порядок диференціального рівняння, а $2s$ -- порядок виразу, який множиться на $\lambda$. Далі припускаєм що
\[
f_t \not= 0, \quad g_s \not= 0, \quad, t>s\ge0
\]
В залежності від того чи буде $s=0$ розглянем два випадки.

До дифеенціального рівняння порядку $2t$ додаються ще $2t$ лінійних однорідних крайових умов
\[
U_\mu [u] = 0 \quad (\mu = 1, 2, \dots, 2t)   \quad (4.14)
\]

\clearpage

\section{Метод послідовних наближень в загальному випадку}Метод послідовних наближень можна застосовувати при дуже загальних припущеннях, а саме, якщо власне значення $\lambda$ входить лінійно в диференціальне рівняння і крайові умови (процес можна застосовувати також для диференціальних рівнянь в часткових похідних). В основу будуть покладені диференціальні рівняння (\emph{TODO}) та крайові умови (\emph{TODO}).

В загальному випадку метод полегає в наступному. Починаючи з довільно вибраної функції $F_0(x)$, визначають послідовність функцій $F_1, F_2, \dots$ при цьому $F_k$ отримують із $F_{k+1}$, шляхом розв'язку крайової задачі. У всіх членах диференціального рівняння и крайових умов, які містять $\lambda$ в якості множника, $\lambda y$ щаміняють на $F_{k-1}$, а в членах, вільних від $\lambda$, заміняють $y$ на $F_k$. Наприклад, для задачі $- u'' = \lambda u; u(1) = 0; \lambda u'(0) = u(0)$ отрамали б
\[
	- F''_k = F_{k-1}; F_k(1) = 0; F'_{k-1}(0) = F_k(0).
\] 
Процес спрощується, якщо $\lambda$ не входить в крайові умови. Тоді, починаючи із $F_0(x)$, функції $F_1, F_2, \dots$ визначають шдяхом розв'язку крайових задач
\[
\left.
\begin{array}{l}
	T[F_k] = S[F_{k-1}],\\
	U_\mu[F_k] = 0
\end{array} \right\}
\quad
(k = 1, 2, \dots)   \quad (12.1)
\]
Можна очікувати, що у випадку збіжності $F_n$ з ростом $n$ все більше приймають вигляд власної функції $u_s$. Тоді $\frac{T[F_n]}{S[F_n]}$ повинне бути приблизно рівним відповідному власному значенню $\lambda_s$. Це відношення залежить ще й від $x$. За наближене власне значення $\Lambda$ приймають величину
\[
\Lambda = \frac{\int\limits_a^b F_n T[F_n]\,\mathrm{d}x}
			   {\int\limits_a^b F_n S[F_n]\,\mathrm{d}x}
		  -
		  \frac{\int\limits_a^b F_n S[F_{n-1}]\,\mathrm{d}x}
			   {\int\limits_a^b F_n S[F_n]\,\mathrm{d}x}
   \quad (12.2)
\]
отриману за способом Релея.

\section{Введення сталих Шварца $a_k$ і відношень $\mu_k$}
Нехай задача на власні значення вигляду (4.11) - (4.14) самоспряжена,виконуються умови повної визначеності (8.2) і власне значення $\lambda$ не входить в крайові умови.

Початкова функція $F_0(x)$ може бути вибрана довільною, але так, щоб виконувались наступні умови: функція повинна бути неперервною разом зі своїми $2n$ похідними. Вона може не задовільняти всіх граничних умов, достатньо, щоб для довільної функції порівняння $u$ мало місце
\[
	\int\limits_a^b (F_0 S[u] - u S[F_0])\,\mathrm{d}x = 0, 
	\int\limits_a^b F_0 S[F_0]\,\mathrm{d}x > 0. (12.3)
\]
З допомогою перетворення Діріхле (4.17) можна в кожному окремому випадку легко встановити, які крайові умови повинна задовільняти $F_0$ для виконання (12.3). Наприклад, у випадку $S[y]=g_0 y$, тобто для окремого виду задачі на власні значення, умови (12.3) виконуються автоматично. Функція $F_0$ не повинна при цьому задовільняти якісь крайові умови і може бути довільною неперервною функцією, тотожно не рівною нулю. Якщо ж саме $F_0$ задовільняє всі $2m$ крайові умови, то внаслідок самоспряженості, зрозуміло, що буде виконуватись і умова (12.3).

Для отримання точних границь вибирають зазвичай $F_0$ так, щоб були вже виконані всі крайові умови. Але інколи для зручності обчислення підкорюють $F_0$ лише деяким крайовим умовам, і тоді лише функції $F_1, F_2, \dots$, визначені в (12.1), задовільняють всі $2m$ крайові умови. (Для практичних цілей часто обмежуються $F_1$.)

\end{document}
