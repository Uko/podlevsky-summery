\documentclass[12pt,a4paper]{article}
\usepackage[english, ukrainian]{babel}
\usepackage[utf8]{inputenc}
\usepackage[T2A]{fontenc}
\usepackage[left=2.5cm,top=2cm,right=1.5cm,bottom=2cm,nohead]{geometry}
\usepackage{setspace}
\usepackage{listings}
\usepackage{color}
\usepackage{float}
\usepackage[pdftex]{graphicx}
\usepackage{courier}
\usepackage{bold-extra}
\usepackage{fix-cm}
\usepackage{alltt}
\usepackage{indentfirst}
\usepackage{amsmath, amsthm, amssymb}
\usepackage{url}

\setstretch{1.1}
\pagestyle{empty}

\begin{document}
\pretolerance=-1
\tolerance=2300

\setlength{\parindent}{1.5cm}
\fontsize{14pt}{6mm}\selectfont

\begin{center}
  Міністерство освіти і науки, молоді та спорту України
  
  Львівський національний університет імені Івана Франка

  Факультет прикладної математики та інформатики
\end{center}


\vspace{6cm}

\begin{center}
  {\bfseries\Large Реферат}\\[0.5cm]
  на тему:\\[0.5cm]
  {\bfseries\Large Сталі $a_k$ і відношення $\mu_k$ Шварца для узагальненої задачі на власні значення}\\
\end{center}

\vspace{2cm}

\begin{flushleft}\leftskip11cm
  Виконали\\
  студенти ПМІ-51м\\
  Михалевич Ігор\\
  Тимчук Юрій
\end{flushleft}

\vspace{6cm}

\begin{center}
  Львів - 2012 
\end{center}

\clearpage

\setstretch{1.5}
\fontsize{14pt}{6mm}\selectfont

\section{Вступ}

Нехай диференціальне рівняння має такий вигляд
\begin{equation}\label{eq:dfEq}
	T[u] = \lambda S[u]
\end{equation}
де $Tu$ та $Su$ лінійні однорідні звичайні диференціальні вирази типу
\begin{equation}\label{eq:defTu}
	T[u] = \sum_{v=0}^t(-1)^v[f_v(x)u^{(v)}(x)]^{(v)}
\end{equation}
\begin{equation}\label{eq:defSu}
	Su = \sum_{v=0}^s(-1)^v[g_v(x)u^{(v)}(x)]^{(v)}
\end{equation}
Тут $(f_v(x)$ та $g_v(x)$ задані, суттєві, $v$-катно неперервно диференційовні функції. 

Нехай $t>s$ тоді, відповідно, $2t$ -- порядок диференціального рівняння, а $2s$ -- порядок виразу, який множиться на $\lambda$. Далі припускаєм що
\[
f_t \not= 0, \quad g_s \not= 0, \quad, t>s\ge0
\]
В залежності від того чи буде $s=0$ розглянем два випадки.

До дифеенціального рівняння порядку $2t$ додаються ще $2t$ лінійних однорідних крайових умов
\begin{equation}\label{eq:dfBoundCond}
	U_\mu [u] = 0 \quad (\mu = 1, 2, \dots, 2t)
\end{equation}

\clearpage

\section{Метод послідовних наближень в загальному випадку}Метод послідовних наближень можна застосовувати при дуже загальних припущеннях, а саме, якщо власне значення $\lambda$ входить лінійно в диференціальне рівняння і крайові умови (процес можна застосовувати також для диференціальних рівнянь в часткових похідних). В основу будуть покладені диференціальні рівняння \eqref{eq:dfEq} та крайові умови \eqref{eq:dfBoundCond}.

В загальному випадку метод полегає в наступному. Починаючи з довільно вибраної функції $F_0(x)$, визначають послідовність функцій $F_1, F_2, \dots$ при цьому $F_k$ отримують із $F_{k+1}$, шляхом розв'язку крайової задачі. У всіх членах диференціального рівняння и крайових умов, які містять $\lambda$ в якості множника, $\lambda y$ щаміняють на $F_{k-1}$, а в членах, вільних від $\lambda$, заміняють $y$ на $F_k$. Наприклад, для задачі $- u'' = \lambda u; u(1) = 0; \lambda u'(0) = u(0)$ отрамали б
\[
	- F''_k = F_{k-1}; F_k(1) = 0; F'_{k-1}(0) = F_k(0).
\] 
Процес спрощується, якщо $\lambda$ не входить в крайові умови. Тоді, починаючи із $F_0(x)$, функції $F_1, F_2, \dots$ визначають шляхом розв'язку крайових задач
\begin{equation}\label{eq:recF_i}
	\left.
	\begin{array}{l}
		T[F_k] = S[F_{k-1}],\\
		U_\mu[F_k] = 0
	\end{array} \right\}
	\quad
	(k = 1, 2, \dots)
\end{equation}
Можна очікувати, що у випадку збіжності $F_n$ з ростом $n$ все більше приймають вигляд власної функції $u_s$. Тоді $\frac{T[F_n]}{S[F_n]}$ повинне бути приблизно рівним відповідному власному значенню $\lambda_s$. Це відношення залежить ще й від $x$. За наближене власне значення $\Lambda$ приймають величину
\[
\Lambda = \frac{\int\limits_a^b F_n T[F_n]\,\mathrm{d}x}
			   {\int\limits_a^b F_n S[F_n]\,\mathrm{d}x}
		  -
		  \frac{\int\limits_a^b F_n S[F_{n-1}]\,\mathrm{d}x}
			   {\int\limits_a^b F_n S[F_n]\,\mathrm{d}x}
   \quad (12.2)
\]
отриману за способом Релея.

\section{Введення сталих Шварца $a_k$ і відношень $\mu_k$}

Нехай задача на власні значення вигляду \eqref{eq:dfEq} - \eqref{eq:dfBoundCond} самоспряжена, виконуються умови повної визначеності
\begin{equation*}
\left. \begin{array}{l}
	(u, T[u]) > 0,\\
	(u, N[u]) > 0
\end{array} \right\}
\quad \text{умови повної визначеності}
\tag{*.1}
\end{equation*}
 (*.1) і власне значення $\lambda$ не входить в крайові умови.

Початкова функція $F_0(x)$ може бути вибрана довільною, але так, щоб виконувались наступні умови: функція повинна бути неперервною разом зі своїми $2s$ похідними. Вона може не задовільняти всіх граничних умов, достатньо, щоб для довільної функції порівняння $u$ мало місце
\begin{equation}\label{fCond}
\begin{array}{l}
	(F_0, S[u]) = (u, S[F_0]),\\
	(F_0, S[F_0]) > 0.
\end{array}
\end{equation}
З допомогою перетворення Діріхле 
\begin{equation*}
\begin{array}{l}
	(u,M[v])-(v,M[u]) =\\
	\left[
	\sum_{\nu=0}^m \sum_{\rho=0}^{v-1}
		(-1)^{\nu+\rho}
		\{
		 u^{(\rho)}[f_{\nu}(x)v^{(\nu)}]^{(\nu-1-\rho)}
		-v^{(\rho)}[f_{\nu}(x)u^{(\nu)}]^{(\nu-1-\rho)}
		\}
	\right]_a^b
\end{array}
\tag{*.2}
\end{equation*}
можна в кожному окремому випадку легко встановити, які крайові умови повинна задовільняти $F_0$ для виконання \eqref{fCond}. Наприклад, у випадку $S[y]=g_0 y$, тобто для окремого виду задачі на власні значення, умови \eqref{fCond} виконуються автоматично. Функція $F_0$ не повинна при цьому задовільняти якісь крайові умови і може бути довільною неперервною функцією, тотожно не рівною нулю. Якщо ж саме $F_0$ задовільняє всі $2t$ крайові умови, то внаслідок самоспряженості, зрозуміло, що буде виконуватись і умова \eqref{fCond}.

Для отримання точних границь вибирають зазвичай $F_0$ так, щоб були вже виконані всі крайові умови. Але інколи для зручності обчислення підкорюють $F_0$ лише деяким крайовим умовам, і тоді лише функції $F_1, F_2, \dots$, визначені в \eqref{eq:recF_i}, задовільняють всі $2t$ крайові умови. (Для практичних цілей часто обмежуються $F_1$.)

За допомогою $F_n$, знайдених таким способом з розв’язку крайової задачі, утворюють, введені Шварцом, сталі $a_k$ (сталі Шварца):
\begin{equation}\label{ShvConst}
	a_k = (F_i, S[F_{k-i}]), \quad 0 \leq i \leq k, \quad k = 0,1,2, \dots
\end{equation}
Вони лише зовнішньо залежать від $i$, так як, використовуючи ітераційне правило \eqref{eq:recF_i} і самоспряженість, можна отримати:
\begin{equation}\label{ShvConstTr}
\begin{array}{l}
	a_k \overset{\eqref{eq:recF_i}}{=} (F_i, T[F_{k-i+1}])
	    \overset{\text{самоспряженість}}{=} (F_{k-i+1}, T[F_i]) \overset{\eqref{eq:recF_i}}{=} \\
	    \overset{\eqref{eq:recF_i}}{=} (F_{k-i+1}, S[F_{i-1}])
	    \overset{\eqref{fCond}}{=} (F_{i-1}, S[F_{k-1+1}])
\end{array}
\end{equation}
тобто скалярний добуток $(F_i, S[F_{k-i}])$ залежить лише від суми $i+(k-i)$, а отже, лише від $k$. Для прикладу,
\begin{equation}\label{ShvConstEx}
	a_2 = (F_2, S[F_0]) = (F_1, S[F_1]) = (F_0, S[F_2])
\end{equation}
В силу (*.1) всі $a_k$ додатні, так як
\begin{equation}\label{ShvConstPosCond}
\begin{array}{l}
	a_{2k} = (F_k, S[F_k]) > 0,\\
	a_{2k-1} = (F_k, T[F_k]) > 0.
\end{array}
\end{equation}
Використовуючи сталі Шварца $a_k$, можна вичислити відношення Шварца
\begin{equation}\label{ShvEq}
	{\mu}_{k+1} = \frac{a_k}{a_{k+1}}, \quad k=0,1,2,\dots
\end{equation}
Враховуючи \eqref{ShvConstPosCond} всі $\mu_k$ додатні. Відношення $\mu_{2k}$ (з парним індексом) можна записати у вигляді відношень Релея:
\begin{equation}\label{RelEq}
	\mu_{2k} = \frac{a_{2k-1}}{a_{2k}} = \frac{(F_k, T[F_k])}{(F_k, S[F_k])} = R[F_k]
\end{equation}

\section{$\mu_k$ утворюють монотонно незростаючу послідовність}

Це легко визначити, якщо взяти до уваги, що зважаючи на (*.1) для довільної функції порівняння u
\[
	Q_1 = (u, T) > 0, \quad Q_2 = (u, S) > 0
\]

Покладемо тепер
\[
	u = a_{2k+1} F_k - a_{2k} F_{k+1}
\]
Ця функція $u$, коли $u \not\equiv 0$, для $k=0$ є напівдопустимою, а для $k>0$ - функцією порівняння. Можливо також, що $u \equiv 0$, наприклад ми будемо писати $Q_1 \geq 0, Q_2 \geq 0$.

Далі, використовуючи \eqref{ShvConst} і \eqref{ShvConstTr}, отримаємо
\[
\begin{array}{l}
	Q_1 = \int_a^b (a_{2k+1} F_k - a_{2k} F_{k+1})(a_{2k+1} T[F_k] - a_{2k} T[F_{k+1}]) \mathrm{d}x = \\
= a_{2k+1}^2 a_{2k-1} - 2 a_{2k} a_{2k+1} a_{2k} + a_{2k}^2 a_{2k+1} = \\
= a_{2k+1}(a_{2k-1} a_{2k+1} - a_{2k}^2) \geq 0 \quad (k=1,2,\dots), \\
Q_2 = a_{2k}(a_{2k} a_{2k+2} - a_{2k+1}^2) \geq 0 \quad (k=0,1,2,\dots).
\end{array}
\]

Зважаючи на \eqref{ShvConstPosCond} можна дві ці нерівності позділити на $a_{2k} a_{2k+1}^2$ або на $a_{2k} a_{2k+1} a_{2k+2}$, як результат будемо мати
\begin{equation}\label{eq:ShvMuDecr}
\left. \begin{array}{l}
		\mu_{2k} \geq \mu_{2k+1} \quad (k=1,2,\dots), \\
		\mu_{2k+1} \geq \mu_{2k+2} \quad (k=0,1,2,\dots).
\end{array} \right\}
\end{equation}
Отже, $\mu_k$ монотонно спадає. Згідно \eqref{RelEq} $\mu_2k$ євідношенням Релея $R[F_k]$. За теоремою про розв’язок варіаційної задачі знаходження мінімуму відношення Релея $R[F_k] \geq \lambda_1$; тому всі $\mu_k$ більші або рівні $\lambda_1$.
Отже, має місце

\textbf{Теорема}. \emph{Нехай задача на власні значення \eqref{eq:dfEq} - \eqref{eq:dfBoundCond} є самоспряженою, виконується умови повної вихначеності (*.1) і власне значення $\lambda$ не входить в крайові умови. Якщо початкова функція $F_0$ задовільняє умові \eqref{eq:defSu}, то відношення Шварца $\mu_k$, обчислене згідно \eqref{eq:recF_i}, \eqref{fCond}, \eqref{ShvEq} з допомогою $F_0$, утворюють монотонну спадаючу послідовність, обмежену знизу першим власним значення $\lambda_1$ і, таким чином, збіжна послідовність 
\begin{equation}\label{eq:ShvLim}
	\mu_1 \geq \mu_2 \geq \mu_3 \geq \dots \geq \lambda_1 .
\end{equation}}
Звідси витікає існування нижньої границі послідовності $\mu_k$, але нічого неможна сказати про величину цієї границі, окрім того, що вона більша або рівна $\lambda_1$. Якщо, наприклад, $F_0(x)$ є $s$-ю власною функцією $y_s$, то $\mu_k = \lambda_s$ для всіх $k$. Границя в цьому випадку рівня $\lambda_s$. Таким чином $\mu_k$ можуть бути дуже віддалені від $\lambda_1$. Для обчислення дуже важливо, що при відомих додаткових умовах $\mu_k$ можуть бути достатньо близькі до $\lambda_1$ і що навіть можна оцінити величину похибки $|\mu_k - \lambda_1|$.

\emph{Приклад}. Знайдемо сталі та відношення Шварца до наступної задачі:
\[
\begin{array}{l}
	-y'' = \lambda (1 + \sin x) y, \\
	y(0) = y(\pi) = 0.
\end{array}
\]
Ми виходимо з функцій, які задовільняють крайовим умовам:
\[
	F_1 = \sin x
\]
і визначаємо $F_0$ згідно \eqref{eq:recF_i} з рівності
\[
	-F_1'' = (1 + \sin x) F_0,
\]
тоді
\[
	F_0 = \frac{\sin x}{1 + \sin x}.
\]
Знайдемо сталі Шварца
\[
\begin{array}{l}
	a_0 = \int_0^{\pi}(1 + \sin x) F_0^2 \mathrm{d}x = \int_0^{\pi} \frac{\sin^2 x}{1 + \sin x} \mathrm{d}x = 4 - \pi, \\
	a_1 = \int_0^{\pi}(1 + \sin x) F_0 F_1 \mathrm{d}x = \int_0^{\pi} \sin^2 x \mathrm{d}x = \frac{\pi}{2}, \\
	a_2 = \int_0^{\pi}(1 + \sin x) F_1^2 \mathrm{d}x = \int_0^{\pi} (1 + \sin x) \sin^2 x \mathrm{d}x = \frac{\pi}{2} + \frac{4}{3}
\end{array}
\]
і відношення
\[
	\mu_1 = \frac{a_0}{a_1} = \frac{8}{\pi} - 2 = 0,54648
	, \quad
	\mu_2 = \frac{a_1}{a_2} = \frac{1}{1 + \frac{8}{3\pi}} = 0,54088
\]

\end{document}
